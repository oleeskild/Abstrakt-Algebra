\documentclass[12pt,a4paper]{article}
\usepackage[norsk]{babel}
\usepackage[utf8]{inputenc}
\usepackage{amsmath}
\usepackage{amsfonts}
\usepackage{amssymb}
\usepackage{eso-pic}
\usepackage{pst-plot}
\usepackage[utf8]{inputenc}
\usepackage{graphicx}
\usepackage{gensymb}
\title{MAT220 HANDBOOK}
\begin{document}
\AddToShipoutPictureBG{%
  \AtPageUpperLeft{%
    \hspace{\paperwidth}%
    \raisebox{-\baselineskip}{%
      \makebox[0pt][r]{Studentnummer: 226286    }
}}}%

\maketitle

\section*{Grupper}
For aa vaere gruppe maa et sett oppfylle 3 aksiomer:\\
$G_1$: For alle $a,b,c \in G$ har vi at $(a*b)*c = a*(b*c)$ (Assositivitet)\\
$G_2$: Det finnes et element e i G slik at for alle $x \in G$ saa $e*x = x*e = x$ (Identitetselementet)
$G_3$: Til hver element $a \in G$ finnes det et elemen a' i G slik at $a*a' = a'*a = e$ (Inversen til a)

Abelian: $ab = ba$\\
Non-abelian: $ab \neq ba$.

\subsection*{Normal Subgroup}
Oppfyller en av disse kravene:\\
La $H \subset G$\\
H er da en normal subgruppe dersom $gHg^{-1} = H$, som er ekvivalent med at\\
$gH = Hg^{-1}$

\subsection*{Factor groups / Quotient Groups}
Dersom en subgruppe er normal vil cosetene dens lage en faktor gruppe. 

En faktorgruppe er ett sett der elementene er sett. Dvs. at hvert element i settet er igjen et sett.

\subsection*{Isomorfisme}
En gruppe $ Z_{m} \times Z_{n} $ er syklisk og isomorfisk til $ Z_{mn} $ bare dersom m og n er relativ primsk.
Altsaa dersom gcd(m, n) er 1. (teorem 11.5, se side 106)\\

To sykliske grupper med samme orden er isomorfisk til hverandre. 
Det samme gjelder for to primtalls grupper av samme orden, der en primtalls gruppe er en gruppe der orderen til gruppen er et primtall. 

\section*{Order}
\subsection*{Order i en gruppe}
Order vil vaere antall elementer i en gruppe.\\
Sagt paa en annen maate; det er antall ganger en maa gaa framover foer en naar identidets elementet. \\
Dermed, naar en skal finne orderen til et element i en gryppe er det du spoer om hvor mange ganger du er noddt til aa utfore gruppeoperasjonen paa denne for aa komme til identietselementet.

Kort om order, fint forklart fra stackexchange:\\
The order of an elements g in a group G is the smallest number of times that you need to apply the group operation to g to obtain the identity.

Let G be cyclic of order 35. That means that there is an element $g \in G$ with $g^{35}=e$, and that $gk\neq e$ for all $1<k<35$. Now, consider $h=g^5.$ Then $h^7=(g^5)^7=g^{35}=e$, but$ hk \neq e $for all $1<k<7$, thus h has order 7. Similarly, the element g7 has order 5.

Remark: Cauchy's theorem (which perhaps you did not see yet) states that if p is a prime dividing |G|, then G has an element of order p. Thus, the only finite groups where all elements except the identity have the same order are p-groups, namely groups whose order is a power of a fixed prime p. A group of size 35 is not a p-group.

\subsection*{Finn order til X i gruppe A}
Formel: $\frac{Order of(A)}{gcd(x, Order of (A)}$

\subsection*{Finn alle abelske grupper av orden X}
Primtallfaktoriser X. Lagranges theorem brukes. Dette sier at orderen til alle subgrupper til G vil kunne dividere orderen til G. \\
Dermed vil alle grupper satt sammen av primtallene til X lage grupper av orden X.\\
Eksempel: Gi alle abelske grupper av orden 8:\\
Primtallfaktoriserer forst $8 = 2 \times 2 \times 2$\\
Alle abelske grupper med order 8 vil da vaere:\\
1. $Z_2 \times Z_2 \times Z_2$\\
2. $Z_4 \times Z_2$\\
3. $Z_8$

\subsection*{Hva er ordenen til (x,y) i $Z_a \times Z_b$}
Finn ordenen til x i $Z_a$. Dvs, $\frac{a}{gcd(x,a)}. $
Finn saa ordenen til y i $Z_b$. Dvs, $\frac{b}{gcd(y,b)}$
Finn til slutt lcm av svarene.\\
Eksempel: $(10,21) i Z_{12} \times Z_{30}$
Orderen til 10 i $Z_12$ vil vaere:\\
$\frac{12}{gcd(10,12)} = \frac{12}{2} = 6$\\
Videre er orderen til 21 i $Z_{30}$ lik:\\
$\frac{30}{gcd(21,30)} = \frac{30}{3} = 10$\\
Orderen til $(10,21)$ i $Z_{12} \times Z_{30}$ er dermed $lcm(10,6) = 30$\\

\section*{Homomorfisme og kernel}
En mapping er homomorfisk dersom $\sigma(ab) = \sigma(a) \sigma(b)$ i multiplikativ notasjon og $\sigma(a+b) = \sigma(a) + \sigma(b)$ i additiv notasjon. \\\\

La G vaere en gruppe som blir mappet til H gjennom $\sigma$\\
Kernel $\sigma$ er da $\{g\in G | \sigma(g) = e_H\}$. Dvs alle elementer i G som mapper til identitets elementet til H.

\section*{Sylow Theoremer}
\subsection*{Theorem 1}
La G vaere en endelig gruppe. Dersom p er et primtall slik at $p^k$ er en divisor for $|G|$ for en $k\geq0$, vil G ha en subgruppe som har order $k^p$. \\
Eks: $S_5$ har order $5! = 5*4*3*2*1 = 5*2*2*3*2 = 5*3*2^3$. Det betyr at den inneholder 3 subgrupper med order 5,3 og $2^3 = 8$.

\subsection*{Definisjon 1}
La G vaere en endelig gruppe og la p vaere et primtall. En subgruppe P av G kalles \"a Sylow p-subgroup\" av G  dersom $|P| = p^k$ for en integer $k\geq 1$ slik at $p^k$ er en divisor for $|G|$ men $p^{k+1}$ er ikke.\\
Eks: Vi tar igjen for oss $|S_5| = 5*3*2^3$. Denne har da en Sylow 2-subgruppe ettersom 2 er det eneste primtallet som opphoyes i noe hoyere enn 1. 

\subsection*{Theorem 2}
La G vaere en endelig gruppe med order n og la p vaere et primtall.\\
Alle Sylow p-subgrupper av G er konjugate (http://en.wikipedia.org/wiki/Conjugacy\_class) og enhver p-subgruppe av G finnes i en Sylow p-subgruppe. 

\subsection*{Theorem 3}
La $n = mp^k$ der $gcd(m,p) = 1$ (dvs. at m og p er relativt primske) og la s vaere antall Sylow p-subgrupper av G. Da vil $s | m $ og $s \equiv 1(mod\  p )$\\
(n maa kunne dele orderen til den originale gruppe, dvs. bestaa av de andre primtallene i G. Dette betyr igjen at m er satt sammen av andre primtall i G.)\\\\

Det holder aa sjekke om primtallene ulike fra p (altsaa de som er relativt primske til p) kan settes sammen til noe som blir 1mod(p). Da vil denne sammensetningen vaere antall p-grupper. 

\subsection*{Folger av dette}
La $p>2$ vaere et primtall, og la G vaere en gruppe med order 2p. Da er G enten syklisk eller isomorfisk til dihedral gruppen $D_p$ av order 2p.\\\\

La G vaere en gruppe av order pq hvor $p>q$ er primtall: \\
Dersom q ikke er en divisor av $p-1$ så er G syklisk. \\
Dersom q er en divisor av $p-1$ saa er enten G syklisk ellers er G generert av to elementer a og b som tilfredstiller disse kravene:\\
$a^p = e$, $b^q = e$, $ba = a^n b$ der $n \neq 1(mod\ p)$ men $n^q \equiv 1 (mod\ p)$.

\subsection*{Fin oppgave for aa se dette i praksis er oppgave 4 i dette settet}
http://bit.ly/1ciHUTD\\
Fasit:
http://bit.ly/1LJ2IjW


\section*{Ringer/Rings}
En ring $< R, +, * >$ er bare et set R som har to operasjone istedenfor bare en (Slik grupper har). Kaller dem i framtidige eksempler for addering og multiplikasjon. For aa vaere en ring maa disse tre aksiomene oppfylles:\\
$R_1$: <R, +> er en abelian gruppe (Det vil si at den maa oppfylle alle gruppe aksiomene under addisjon).\\
$R_2$: Multiplikasjon er assosiativt.\\
$R_3$: For alle $a,b,c \in R$ saa holder $a*(b+c) = (a*b) + (a*c)$ og $(a+b) * c = (a*c) + (b*c)$.

Legg merke til at en ring IKKE nodvendigvis trenger aa oppfylle folgende krav\\
1) Multiplisering kommutativ, dvs: $a*b = b*a$.\\
2) Multipliserings identitet\\
3) Multipliserings inverser.\\

\subsection*{Subring}
For aa bevise at S er en subring av R maa en bevise folgende:\\
0) Vise at S faktisk er et subset av R (Er ofte obvious).\\
1) S er lukket under addisjon.\\
2) S er lukket under multiplikasjon.\\
3) S inneholder det adderende identitetselementet (0).\\
4) Alle elementer i S har en invers i S.

Protip: For aa bevise at noe er en ring kan en ofte kun bevise det som trengs for aa bevise at det er en subring, og dersom det da er et subset av noe en vet er en ring er det nok aa bevise.

\subsection*{Zero divisors og Integral domain}
Zero divisor er tall ulik 0 som, naar de multipliseres, er lik 0. Eks:\\
Hvis vi ser paa $Z_{10}$ saa vil $5 \times 4$ vaere lik 20mod(10) $= 0$. Da, til tross for at hverken 5 eller 4 er 0 er resultatet 0. 5 og 4 er da, sammen, zero divisors i $Z_{10}$. \\

Videre finnes det ringer der dette aldri skjer. (F.eks. R, Q og Z). Her kan aldri to elementer multiplisert sammen bli 0. Disse ringene kalles INTEGRAL DOMAIN. Dette er den offisielle definisjonen:\\
En kommutativ ring med enhetselement og ingen zero divisors.

Subnote: Legg ogsaa merke til at alle $Z_c$ der c ikke er et primtall har zero divisors, mens $Z_p$ der p er et primtall er et integral domain. 

\section*{Field (Kropp)}
Et field er bare et integral domain hvor hvert non-zero element er en unit. En unit er hvilket som helst element som har en invers (multipliserende invers that is). \\
Eksempler paa field er R og Q. Da disse oppfyller alle integral domain kravene, men ogsaa oppfyller kravet om at alle elementer har en invers (sett bort ifra 0). Z er ikke et field, da denne ikke inneholder noen multiplikativ invers for sine elementer. \\

Protip(theorem): Et endelig integral domain er alltid ett field. \\
Protip 2: La R vaere en ring og I vaere et maksimalt ideal for denne ringen. R/I er da en kropp/field. 

\subsection*{Adjoint Field}
Et adjoint field er naar en allerede har et ferdig field, si Q, og oensker aa legge til et nytt element, si $\sqrt{2}$, men onsker at det forsatt skal forbli et field. Da maa man ogsaa legge til flere elementer slik at aksiomene for field ikke brytes.\\
I dette eksempelet maa en f.eks. sorge for at alle tall som allerede er i Q skal kunne adderes me $\sqrt{2}$ og fortsatt eksistere i settet.\\
I tillegg vil vi ogsaa at alle tall i Q skall kunnes multipliseres me $\sqrt{2}$ og likevel eksistere i settet. Disse to egenskapene kan vi kombinere og vi faar da settet:\\
$Q(\sqrt{2}) = \{a + b\sqrt{2} | a,b \in Q\}$.\\
Legg merke til at dette ikke er en generell losning. Hvis vi ser på $Q(\sqrt[3]{5})$ ser vi at $\sqrt[3]{5} * \sqrt[3]{5} = \sqrt[3]{5} ^ 2 $ som vi ikke kan uttrykke med elementer fra Q.(gitt at vi bare kan multiplisere og addere). Dermed maa denne ogsaa legges til:\\
$Q(\sqrt[3]{5}) = \{a+b\sqrt[3]{5} + c \sqrt[3]{5}^2\}$  \\
Generelt: Legg til nye ledd (Multipliser med en konstant), der hvert ledd er tallet som skal legges til i fielden opphøyd i en storre enn den forrige, helt til dette tallet blir noe som allerede finnes i fielden. 

\section*{Polynomer og saant}

\subsection*{Vis at polynomet p er irreducible over Q}
Gauss' Lemma forteller oss for å vise at et polynom er irreducible over Q er det nok å vise at det er irreducible over Z. => Vis at p er irreducible over Z, da vil det vaere nok bevis for Q.
\\\\
Eisenstein's criterion: Polynomet er irreducibel dersom disse tre kravene er oppfyllt (dette vil som regel vaere den som brukes oftest)\\
Dersom vi kan finne et primtall integer q slik at:\\
1. q er en faktor for hver ikke ledende koeffisient (dvs. alle konstanter som ikke er foran den x'en med høyest opphøyning. I $5x^4 +2x -3$ er f.eks. 5 den ledende koeffesienten)\\
2. q er IKKE en faktor i den ledende koeffisienten\\
3. $q^2$ er ikke en faktor i konstanten. \\\\

Reduksjon mod(n): Dersom en modulerer polynomet over en viss n og viser at de da er irredusibelt over $Z_n$ vil det bety at det ogsaa er irreducibelt over Z.\\
Eks: $t^4 - 15t + 7\ mod(5) = t^4 + 2$. Dette skal da kunne reduseres til to faktorer der den ene er polynom grad 1 og det andre med grad 3.(Setter t utenfor) Dvs: $(t-x)(t^3 ....)$\\
Trenger da bare å sjekke for alle tall a i $Z_5$. $x\in Z_5$ er en rot. Dersom ingen av disse blir 0 finnes det ikke en rasjonell rot (det betyr et rasjonalt tall for x slik at polynomet = 0). Ser raskt at det ikke finnes en rot i $Z_5$ for $t^4 + 2$. Men dette beviser kun at det ikke kan faktorieser på denne måten.(Det hadde vaert nok dersom polynomet var av grad 3) Det kan fortsatt faktoriseres med to polynomer med grad 2: \\
$(t^2....)(t^2...)$. Denne er litt mer tricky. For aa finne ut dette bruk metoden som blir beskrevet i denne videoen omtrent fra minutt 12\\
https://youtu.be/ebwp4eqrOfg?t=12m32s

\subsection*{Intuitiv forstaaelse av irreducible}
At et polynom er irreducible i en viss gruppe betyr simpelten at losningen for x ikke finnes i gruppen. F.eks. hvis en ser paa $x^2 - 3$ i Q[x] kan en raskt se at $\sqrt{3}$ er losningen, noe som ikke er et rasjonalt tall (Q[x]). Dermed er det irreducible. 

\subsection*{Vis at polynom p(x) er irreducible i $Z_y$}
Dersom $p(a) \neq 0$ for alle $a \in Z$ er p irredusibelt. \\
(Dersom en modulerer polynomet med en integer m er det nok å vise at formelen ovenfor da stemmer med dette nye polynomet i $Z_m$.


\subsection*{Er p(x) en generator for et maksimal ideal for Q[x]}
Dersom p(x) er irredusibel i Q[x] betyr dette at at p(x) er en generator for et maksimal ideal. 


\section*{Burnsides formel}
Basically, det den gjor, er aa finne antall orbits i en gruppe.
Dersom G er en gruppe som inneholder ulike rotasjoner som: fiksert, roter 90 grader, roter 180, roter 170 grader, speil horisontal, vertikalt, begge diagonale. For hver av disse rotasjonene maa en finne hvor mange elementer totalt som forblir fiksert naar en gjor disse operasjonene. En tar saa summen av dette og deler paa alle de ulike rotasjonene. Veldig godt forklart i denne korte videoen:\\
https://youtu.be/wdDF7\_vfLcE\\\\
Formel som kan brukes:\\
Gitt ett alfabet med n bokstaver, og ett sirkelord med lengde m så finner en antall forskjellige sirkelord ved å først finne gcd til alle tall i $Z_m$ og m. Så dersom m=3 så må en finne gcd(0,3) = x, gcd(1,3) = y, gcd(2,3)=z. Deretter setter en disse verdiene som opphoyere for hver sin n, i denne:\\
$\frac{1}{m}(n^x + n^y + n^z)$. 

\section*{Fermats teorem}
Dersom p er et primtall som ikke deler et heltall n vil p dele: $n^{p-1} - 1$

\end{document}